\listfiles

%% to compile cover
\makeatletter
\newcommand{\makecoverfile}[0]{
  \immediate\write18{latexmk -pdf cover-ua.tex}
}
\makeatother

%% to select input encoding
\usepackage[utf8]{inputenc}
\usepackage[T1]{fontenc}

%% to use internationalization and localization languages
\usepackage[portuguese,french,english]{babel}

%% to add and customise hyperlinks
\usepackage[hidelinks,breaklinks=true,colorlinks=true]{hyperref}
\hypersetup{urlcolor=black}
\ifthenelse{\boolean{draft}}{
        \hypersetup{linkcolor=blue,citecolor=blue}%
    }{
        \hypersetup{linkcolor=black,citecolor=black}%
    }

%% to put draft watermark
\ifthenelse{\boolean{draft}}{\usepackage{draftwatermark}}{}

%% to merge pdfs
\usepackage{pdfpages}

%% to set page margins
\usepackage[left=3cm,right=2.5cm,bottom=3cm,top=3cm]{geometry}
\ifthenelse{\boolean{margins}}{\geometry{showframe}}{}

%% to add frame aroung images
\usepackage[export]{adjustbox}
\ifthenelse{\boolean{frames}}{
    \let\includegraphicsold\includegraphics%
    \renewcommand{\includegraphics}[2][]{\adjustbox{frame}{\includegraphicsold[#1]{#2}}}
    }{}

%% to refine text
\usepackage{microtype}

%% to typeset quantities in consistent way
\usepackage{siunitx}
\sisetup{detect-all,
         list-final-separator={, and },
         list-units=single,  % repeat or single
         detect-weight=true,
         detect-inline-weight=math,
         table-text-alignment = right,
         table-number-alignment = right,
         product-units = power,
         }
\DeclareSIUnit\pixel{px}
\DeclareSIUnit\greylevel{gl}

%% to use bold math symbols
\usepackage{bm}

%% to add dummy text
\usepackage{lipsum}

%% to add appendices
\usepackage{appendix}

%% to use international system units
\usepackage{siunitx}

%% to add inline TT text (e.g. code snippets)
\usepackage{verbatim}

%% to add frames around figures and allow force placement
\usepackage{float}

%% to enable floats on facing pages
\usepackage{dpfloat}

%% to add pdf comments
\usepackage{comment}
\usepackage{pdfcomment}

%% to control hyphenation 
% \hyphenpenalty=100
\doublehyphendemerits=1000000

%% to customise bookmarks
\usepackage[open,openlevel=0,numbered]{bookmark}

%% to enhance the quality of tables
\usepackage{booktabs}

%% to create table layouts
\usepackage{makecell}

%% to customise tables row spacing
\renewcommand{\arraystretch}{1.25} % default : 1.0

%% to use tables with footnotes
\usepackage[normal]{threeparttable}

%% to add colors to tables
\usepackage{colortbl}

%% to use text in multiple columns
\usepackage{multicol}

%% to use math symbols
\usepackage{amsmath}
\usepackage{amssymb}

%% to extend amsmath package
\usepackage{mathtools}

%% to use an alternative interface to graphics
\usepackage{graphicx}
\ExecuteOptions{draft}
%
% to set the main path of figures
\graphicspath{{figures/}}

%% to use advanced colors
\usepackage{xcolor}
%
% to set the color of dem
\ifthenelse{\boolean{lesscolor}}{
        \definecolor{dem}{cmyk}{0,0,0,1}               % if lesscolor option active turn black
    }{
        \definecolor{dem}{cmyk}{0.26,0.90,0.94,0.24}
    }

\newcommand{\colordem}[1]{\textcolor{dem}{#1}}
\newcommand{\colorlink}[1]{\textcolor{dem}{#1}}
%
% to set gray highlight color 
\definecolor{highlight}{cmyk}{0.0,0.0,0.0,0.3}

%% to use colors in tables
\usepackage{colortbl}

%% to use advanced quotes
\usepackage{csquotes}

%% to customise lists
\usepackage{enumitem}
\setlist[enumerate]{itemsep=0pt,partopsep=0pt}

%% to capitalize first word of text
\usepackage{mfirstuc}

%% to use extended array–and tabular–environments
\usepackage{array}

%% to add lof and lot to toc
\usepackage[nottoc,notbib,notlof,notlot]{tocbibind}

%% to use independent subfiles
\usepackage{subfiles}
\newboolean{InSubfile}
\setboolean{InSubfile}{true}

%% to linebreak url at appropriate character 
\usepackage{url}
\urlstyle{rm}

%% to cite labels from another subfile
\usepackage{xr}
\usepackage{zref}

\makeatletter
\newcommand*{\addFileDependency}[1]{% argument=file name and extension
  \typeout{(#1)}
  \@addtofilelist{#1}
  \IfFileExists{#1}{}{\typeout{No file #1.}}
}
\makeatother

\newcommand*{\myexternaldocument}[1]{%
    \externaldocument{#1}%
    \addFileDependency{#1.tex}%
    \addFileDependency{#1.aux}%
}

%% to select fonts with lualatex
\usepackage{fontspec}
%
% to set main font
\setmainfont{STIX Two Text}[
    Extension = {.otf},
    UprightFont = {*-Regular},
    BoldFont = {*-Bold},
    ItalicFont = {*-Italic},
    BoldItalicFont = {*-BoldItalic},
    FontFace = {sb}{\shapedefault}{*-SemiBold},
    FontFace = {mb}{\shapedefault}{*-Medium},
]
%
% to set sans serif font
\setsansfont{Helvetica Neue}[
    Extension = {.otf},
    UprightFont = {*-Roman},
    BoldFont = {*-Bold},
    ItalicFont = {*-Italic},
    BoldItalicFont = {*-BoldItalic},
    FontFace = {ul}{\shapedefault}{*-UltraLight},
    FontFace = {el}{\shapedefault}{*-Thin},
    FontFace = {l}{\shapedefault}{*-Light},
    FontFace = {sb}{\shapedefault}{*-Medium},
    FontFace = {eb}{\shapedefault}{*-Heavy},
    FontFace = {ub}{\shapedefault}{*-Black},
]
%
% to declare font series
\DeclareRobustCommand{\ulseries}{\fontseries{ul}\selectfont}
\DeclareRobustCommand{\elseries}{\fontseries{el}\selectfont}
\DeclareRobustCommand{\lseries}{\fontseries{l}\selectfont}
\DeclareRobustCommand{\mbseries}{\fontseries{mb}\selectfont}
\DeclareRobustCommand{\sbseries}{\fontseries{sb}\selectfont}
\DeclareRobustCommand{\ebseries}{\fontseries{eb}\selectfont}
\DeclareRobustCommand{\ubseries}{\fontseries{ub}\selectfont}
%
% to select math font with lualatex
\usepackage[math-style=TeX,bold-style=TeX]{unicode-math}
%
% to set main math font
\setmathfont[version=default]{STIX Two Math}
%
% to define a secondary sans serif math font
\setmathfont[version=sansmath]{Helvetica Neue}
%
% to select current math font
\mathversion{default}

%% to customise toc, lof, and lot
\usepackage[titles]{tocloft}
%
% to remove page style from toc, lof, lot first pages
\tocloftpagestyle{plain}
%
% to add up to subsections in toc
\setcounter{tocdepth}{2}
%
% to customise font style of toc, lof, and lot
\newcommand{\cftfontsize}{\normalsize}
\renewcommand{\cftchapfont}{\sffamily\bfseries\cftfontsize}
\renewcommand{\cftsecfont}{\sffamily\cftfontsize}
\renewcommand{\cftsubsecfont}{\sffamily\cftfontsize}
\renewcommand{\cftfigfont}{\sffamily\cftfontsize}
\renewcommand{\cfttabfont}{\sffamily\cftfontsize}
%
% to put page number after section, subsection, figure, and table
\renewcommand{\cftdot}{}
\renewcommand{\cftchapleader}{\hfill}
\renewcommand{\cftsecleader}{\hspace{1em}}
\renewcommand{\cftsecafterpnum}{\cftparfillskip}
\renewcommand{\cftsubsecleader}{\hspace{1em}}
\renewcommand{\cftsubsecafterpnum}{\cftparfillskip}
\renewcommand{\cftfigleader}{\hspace{1em}}
\renewcommand{\cftfigafterpnum}{\cftparfillskip}
\renewcommand{\cfttableader}{\hspace{1em}}
\renewcommand{\cfttabafterpnum}{\cftparfillskip}
\renewcommand{\cftpnumalign}{l}
%
% to put chapter page numbers aligned right
\makeatletter
\if@cfthaschapter
  \renewcommand{\cftchapfillnum}[1]{%
    {\cftchapleader}\nobreak%
    \makebox[\@pnumwidth][r]{\cftchappagefont #1}\cftchapafterpnum\par%
  }
\makeatother
%
% to set indents from left margin of figure, table, chapter, section and subsection entries
\cftsetindents{figure}{3ex}{5ex}
\cftsetindents{table}{3ex}{5ex}
\cftsetindents{chapter}{0pt}{3ex}
\cftsetindents{section}{3ex}{5ex}
\cftsetindents{subsection}{8ex}{7ex}
%
% to increase vspace between sections and subsections
\setlength\cftbeforechapskip{2.5ex}
\setlength\cftbeforesecskip{0.5ex}
\setlength\cftbeforesubsecskip{0.5ex}
\setlength\cftbeforefigskip{0.5ex}
\setlength\cftbeforetabskip{0.5ex}
\setlength\cftbeforetoctitleskip{0pt}
\setlength\cftbeforeloftitleskip{0pt}
\setlength\cftbeforelottitleskip{0pt}
\setlength\cftaftertoctitleskip{0pt}
\setlength\cftafterloftitleskip{0pt}
\setlength\cftafterlottitleskip{0pt}
%
\newcommand{\nonumchaptoc}[1]{\hspace{3ex}#1}
\newcommand{\nocontentsline}[3]{}
\newcommand{\tocless}[2]{\bgroup\let\addcontentsline=\nocontentsline#1{#2}\egroup}

%% to use advanced programming tools
\usepackage{etoolbox}

%% to use and customise acronyms
\usepackage[nonumberlist,acronym,nomain,nopostdot]{glossaries}
% Commands
% \gls      : What You See is What You Get (WYSIWYG) and subsequent WYSIWYG
% \acrfull  : What You See is What You Get (WYSIWYG)
% \acrlong  : What You See is What You Get
% \acrshort : WYSIWYG
% \glspl    : WYSIWYGs

% List of Acronyms
% #
% A
% B
% C
% D
% E
% F
% D
% H
% I
% J
% K
% L
% M
% N
% O
% P
% Q
% R
% S
% T
% U
% V
% W
\newacronym{wysiwyg}{WYSIWYG}{what you see is what you get}
% X
% Y
% Z




\glsdisablehyper
\makeglossaries

%% to customise floats caption
\usepackage{subcaption}
\usepackage[font={sf,small},labelfont=bf]{caption}
%
% to define caption options
\captionsetup[table]{labelsep=quad,labelfont=bf}
\captionsetup[figure]{labelsep=quad}

%% to use biblatex to manage bibliography
\usepackage[style=authoryear,citestyle=authoryear]{biblatex}
% Citation Commands
% \cite - Goossens et al. 1994
% \parencite - (Goossens et al. 1994)
% \textcite - Goossens et al. (1994)

% to load bib files
% \bibliography{../bib/phd-thesis.bib}
\addbibresource{../bib/chapter-1.bib}
% \addbibresource{../bib/chapter-X.bib}

% set package options
\ExecuteBibliographyOptions{sortcites=false}
\ExecuteBibliographyOptions{doi=false,isbn=false,url=false}
\ExecuteBibliographyOptions{giveninits=true,terseinits=true}
\ExecuteBibliographyOptions{minnames=1,maxbibnames=12,maxcitenames=2}
\ExecuteBibliographyOptions{uniquelist=false,uniquename=false}

% set bib month entries
\DefineBibliographyStrings{english}{january=January,february=February,march=March,april=April,may=May,june=June,july=July,august=August,september=September,november=November,december=December}

% to use last name, first name
\DeclareNameAlias{author}{family-given}

% to customise citations appearance
\renewcommand*{\citesetup}{\scshape\biburlsetup\frenchspacing}

% to customise item separation lenght
\setlength\bibitemsep{3.0pt plus 2.0pt minus 1.0pt}

% to remove period between authors initials
\renewcommand*{\revsdnamepunct}{}
\renewcommand*{\newblockpunct}{\addcomma\space}

% to add hyperef to doi
\newbibmacro{hyperdoi}[1]{\iffieldundef{doi}{#1}{\href{http://dx.doi.org/\thefield{doi}}{#1}}}
\newbibmacro{hyperurl}[1]{\iffieldundef{url}{#1}{\href{\thefield{url}}{#1}}}
\newbibmacro{hyperlink}[1]{\iffieldundef{doi}{\iffieldundef{url}{#1}{\href{\thefield{url}}{#1}}}{\href{http://dx.doi.org/\thefield{doi}}{#1}}}

\DefineBibliographyStrings{english}{editor = {(Ed.)}, editors = {(Eds.)}}

% to define bib entries and undefined mandatory bib fields
\newcommand{\bibmiss}[2]{#2{\color{red}#1}}
% draft document (comment lines below for final document)
% \newcommand{\bibbook}{\iffieldundef{title}{\bibmiss{Book}{. }{.}}{\printfield{title}}}
% \newcommand{\bibpaper}{\iffieldundef{title}{\bibmiss{Paper}{. }}{\printfield{title}}}
% \newcommand{\bibarticle}{\iffieldundef{title}{\bibmiss{Article}{. }}{\printfield{title}}}
% \newcommand{\bibchapter}{\iffieldundef{title}{\bibmiss{Chapter}{. }}{\printfield{title}}}
% \newcommand{\bibwebpage}{\iffieldundef{title}{\bibmiss{Webpage}{. }}{\printfield{title}}}
% \newcommand{\bibsoftware}{\iffieldundef{title}{\bibmiss{Software}{. }}{\printfield{title}}}
% %
% \newcommand{\bibauthor}{\ifnameundef{author}{\bibmiss{Author}{}}{\printnames{author}}}
% \newcommand{\bibeditor}{\ifnameundef{editor}{\bibmiss{Editor}{}}{\printnames{editor}}}
% \newcommand{\bibpublisher}{\iflistundef{publisher}{\bibmiss{Publisher}{. }}{\printlist{publisher}}}
% \newcommand{\bibinstitution}{\iflistundef{institution}{\bibmiss{Institution}{, }}{\printlist{institution}}}
% \newcommand{\bibjournal}{\iffieldundef{journaltitle}{\bibmiss{Journal}{. }}{\printfield{journaltitle}}}
% \newcommand{\bibproc}{\iffieldundef{booktitle}{\bibmiss{Proceedings}{. }}{\printfield{booktitle}}}
% \newcommand{\bibcollection}{\iffieldundef{booktitle}{\bibmiss{Book}{. }}{\printfield{booktitle}}}
% \newcommand{\bibseries}{\iffieldundef{series}{\bibmiss{Series}{}}{\printfield{series}}}
% %
% \newcommand{\bibpages}{\iffieldundef{pages}{\bibmiss{Pages}{:}}{\printfield{pages}}}
% \newcommand{\biblocation}{\iflistundef{location}{\bibmiss{Location}{, }}{\printlist{location}}}
% \newcommand{\bibyear}{\iffieldundef{year}{\bibmiss{Year}{, }}{\printfield{year}}}
% \newcommand{\bibtype}{\iffieldundef{type}{\bibmiss{Type}{, }}{\printfield{type}}}
% \newcommand{\biburl}{\iffieldundef{url}{\bibmiss{URL}{. }}{\printfield{url}}}
% \newcommand{\bibversion}{\iffieldundef{version}{\bibmiss{Version}{. }}{\printfield{version}}}
% \newcommand{\bibvolume}{\iffieldundef{volume}{\bibmiss{Volume}{, }}{\printfield{volume}}}
% \newcommand{\bibdays}{\iffieldundef{volume}{\bibmiss{Days}{, }}{\printfield{volume}}}
% \newcommand{\bibmonth}{\iffieldundef{month}{}{\printfield{month}}}
% \newcommand{\bibwebsite}{\iffieldundef{titleaddon}{\bibmiss{website}{ }}{\printfield{titleaddon}}}
% \newcommand{\bibbookpages}{\iffieldundef{pages}{\bibmiss{Pages}{}}{\printfield{pages}}}
% %
% final document (comment lines below for draft document)
\newcommand{\bibbook}{\printfield{title}}
\newcommand{\bibpaper}{\printfield{title}}
\newcommand{\bibarticle}{\printfield{title}}
\newcommand{\bibchapter}{\printfield{title}}
\newcommand{\bibwebpage}{\printfield{title}}
\newcommand{\bibsoftware}{\printfield{title}}
%
\newcommand{\bibauthor}{\printnames{author}}
\newcommand{\bibeditor}{\printnames{editor}}
\newcommand{\bibpublisher}{\printlist{publisher}}
\newcommand{\bibinstitution}{\printlist{institution}}
\newcommand{\bibjournal}{\printfield{journaltitle}}
\newcommand{\bibproc}{\printfield{booktitle}}
\newcommand{\bibcollection}{\printfield{booktitle}}
\newcommand{\bibseries}{\printfield{series}}
%
\newcommand{\bibpages}{\printfield{pages}}
\newcommand{\biblocation}{\printlist{location}}
\newcommand{\bibyear}{\printfield{year}}
\newcommand{\bibtype}{\printfield{type}}
\newcommand{\biburl}{\printfield{url}}
\newcommand{\bibversion}{\printfield{version}}
\newcommand{\bibvolume}{\printfield{volume}}
\newcommand{\bibdays}{\printfield{volume}}
\newcommand{\bibmonth}{\printfield{month}}
\newcommand{\bibwebsite}{\printfield{titleaddon}}
\newcommand{\bibbookpages}{\printfield{pages}}

\renewcommand{\mkbibnamefamily}[1]{\textsc{#1}}
\renewcommand{\mkbibnamegiven}[1]{\textsc{#1}}

% to customise article entry
\DeclareFieldFormat[article]{title}{\makefirstuc{#1}}
% \DeclareFieldFormat[article]{author}{\textsc{#1}}
\DeclareFieldFormat[article]{number}{(#1)}
\DeclareFieldFormat[article]{pages}{#1}
\DeclareBibliographyDriver{article}{%
    \usebibmacro{hyperlink}{%
        \bibauthor\newunit%
        \bibarticle\newunit%
        \bibjournal\newblock%
        \bibvolume%
        \iffieldundef{number}{}{\printfield{number}}\printunit{\addcolon}%
        \bibpages\newblock%
        \bibyear%
        \finentry%
    }
}

% to customise inproceedings entry
\DeclareFieldFormat[inproceedings]{title}{\makefirstuc{#1}}
\DeclareFieldFormat[inproceedings]{volume}{#1} % alias for conference days
\DeclareBibliographyDriver{inproceedings}{%
    \usebibmacro{hyperlink}{%
        \bibauthor\newunit%
        \bibpaper\newunit%
        \bibproc\newblock%
        \biblocation\newblock%
        \bibdays\printunit{\space}%
        \bibmonth\newblock%
        \bibyear%
        \finentry%
    }
}

% to customise book entry
\DeclareBibliographyDriver{book}{%
    \usebibmacro{hyperlink}{%
        \bibauthor\newunit%
        \bibbook\newunit%
        \bibpublisher\newblock%
        \biblocation\newblock%
        \bibyear%
        \finentry%
    }
}

% to customise software entry
\DeclareFieldFormat[software]{version}{v#1}
\DeclareBibliographyDriver{software}{%
    \usebibmacro{hyperlink}{%
        \bibauthor\newunit%
        \bibsoftware\newunit%
        \bibversion\newblock%
        \bibyear%
        \finentry%
    }
}

% to customise thesis entry
\DeclareFieldFormat[thesis]{title}{\textit{#1}}
\newcommand{\bibthesis}{\iffieldundef{title}{\bibmiss{Thesis}{. }}{\printfield{title}}}
\DeclareBibliographyDriver{thesis}{%
    \usebibmacro{hyperlink}{%
        \bibauthor\newunit%
        \bibthesis\newunit%
        \bibtype\newblock%
        \bibinstitution\newblock%
        \bibyear%
        \finentry%
    }
}

% to customise thesis entry
\DeclareFieldFormat[online]{title}{\textit{#1}}
\DeclareFieldFormat[online]{url}{\url{#1}}
\DeclareFieldFormat[online]{titleaddon}{#1}
\DeclareBibliographyDriver{online}{%
    \usebibmacro{hyperlink}{%
        \bibauthor\newunit%
        \bibwebpage\newunit%
        \bibwebsite\newblock%
        \biburl\newblock%
        \bibyear%
        \finentry%
    }
}

% to customise incollection entry
\DeclareFieldFormat[incollection]{title}{#1}
\DeclareFieldFormat[incollection]{booktitle}{\textit{#1}}
\DeclareFieldFormat[incollection]{volume}{Volume #1}
\DeclareNameAlias{editor}{family-given}
\DeclareBibliographyDriver{incollection}{%
    \usebibmacro{hyperlink}{%
        \bibauthor\newunit%
        \bibchapter\newunit%
        \usebibmacro{in:}
        \printnames{editor}
        \ifthenelse{\value{editor}>1}{\bibstring{editors}.}{\bibstring{editor}.}
        \bibcollection\newblock%
        \bibvolume\newunit%
        \bibseries\newblock%
        \bibpublisher\newblock%
        \biblocation\newblock%
        \bibyear%
        \finentry%
    }
}

% to customise collection entry
\DeclareBibliographyDriver{collection}{%
    \usebibmacro{hyperlink}{%
        \printnames{editor}
        \ifthenelse{\value{editor}>1}{\bibstring{editors}.}{\bibstring{editor}.}
        \bibbook\newunit%
        \bibpublisher\newblock%
        \biblocation\newblock%
        \bibyear%
        \finentry%
    }
}

% to remove "and" from last author in bib mode and add "and" in citation mode
\DeclareDelimFormat[bib]{finalnamedelim}{\addcomma\space}
\DeclareDelimFormat[cite]{finalnamedelim}{\space and \space}

% to sort bib entries of same first author by year
\DeclareSourcemap{
  \maps[datatype=bibtex]{
    \map{\step[fieldsource=author, match=\regexp{(.*?)\s+and\s+}]
         \step[fieldset=sortname, fieldvalue={$1}]}}}

%% to format sections, subsections, subsubsections titles
\usepackage[explicit,compact,aftersep,clearempty]{titlesec}
%
\newcommand{\chapterstyle}{\filleft\sffamily\Huge\bfseries}
%
\titleformat{\section}{\sffamily\large\bfseries}{\colordem{\thesection}}{1em}{#1}
\titleformat{\subsection}{\sffamily\normalsize\bfseries}{\colordem{\thesubsection}}{1em}{#1}
\titleformat{\subsubsection}{\rmfamily\normalsize\bfseries}{}{}{#1}
\titleformat{\paragraph}[runin]{\rmfamily\normalsize\bfseries}{}{}{#1}
%
\titlespacing*{\chapter}      {0pt}{80pt}{100pt}
\titlespacing*{\section}      {0pt}{2.50ex plus .2ex minus .2ex}{1.6ex}
\titlespacing*{\subsection}   {0pt}{2.25ex plus .2ex minus .2ex}{1.1ex}
\titlespacing*{\subsubsection}{0pt}{2.25ex plus .2ex minus .2ex}{1.1ex}
\titlespacing*{\paragraph}    {0pt}{2.25ex plus .2ex minus .2ex}{1em}

% \titlespacing*{\chapter}      {0pt}{80pt}{100pt}

%% to customise headers and footers
\usepackage{fancyhdr}
%
% to create page style
\renewcommand{\chaptermark}[1]{\markboth{\itshape#1}{}}
\renewcommand{\sectionmark}[1]{\markright{\itshape\thesection~ ~#1}}
\fancypagestyle{mypagestyle}{%
    \fancyhf{}%
    \fancyhead[LE]{\thepage\hspace{20pt}\nouppercase{\itshape\leftmark}}%
    \fancyhead[RO]{\nouppercase{\itshape\rightmark}\hspace{20pt}\thepage}%
    \fancyfoot[CE,CO]{}%
    \fancyfoot[LE,RO]{}%
    \renewcommand{\headrulewidth}{0pt}%
    \setlength{\headheight}{13.59999pt}
}
%
% to remove page number from chapter pages
\fancypagestyle{plain}{
    \renewcommand{\headrulewidth}{0pt}%
    \fancyhf{}%
}
%
% to set page style
\pagestyle{mypagestyle}